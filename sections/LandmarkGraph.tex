\section[L. Graph]{Landmark Graph}
\begin{frame}{Phase A $\sim$ Definition and Recognition of Landmarks}
    \begin{itemize}
        \item $\forall$ landmark $\exists$ 3 features:
            \begin{itemize}
                \item \textbf{Distinctiveness}: Unique patterns distinguishable from surroundings;
                \item \textbf{Stability}: It doesn't change dynamically over time;
                \item \textbf{Identifiability}: Detectable by one or more sensors.
            \end{itemize}
        \item Mathematical definition: \( v = \langle (x, y), (R_1, \ldots, R_M) \rangle \)
            \begin{itemize}
                  \item $(x, y)$: coordinate of the landmark;
                  \item $(R_1, \dots ,R_M)$: detection rule in different types of sensor readings;
                  \item $M$ is the number of rules that this landmark possesses.
            \end{itemize}
    \end{itemize}
\end{frame}

\begin{frame}{Types of Landmarks}
    \begin{columns}
        \begin{column}{.65\textwidth}
            \begin{itemize}[<+->]
                \item \textbf{Accelerometer Landmark}: Detects changes in \textbf{motion state};
                \item \textbf{Gyroscope Landmark}: Detects changes in \textbf{walking direction} using magnetometer \& gyroscope;
                \item \textbf{Barometer Landmark}: Measures \textbf{air pressure}, which changes with \textbf{altitude/height};
                \item \textbf{WiFi Landmark}: Detects location point that overhears the \textbf{strongest RSS from an AP};
                \item \textbf{Light Landmark}: Detects changes in \textbf{light intensity} using light sensor;
            \end{itemize}
        \end{column}%
        \hfill%
        \begin{column}{.35\textwidth}
            \begin{figure}[ht]
                \centering
                \includegraphics<1>[width=\linewidth]{images/racc.jpg}
                \includegraphics<2>[width=\linewidth]{images/rgyro.jpg}
                \includegraphics<3>[width=\linewidth]{images/rbaro.jpg}
                \includegraphics<4>[width=\linewidth]{images/rwifi.jpg}
                \includegraphics<5>[width=\linewidth]{images/rlight.jpg}
            \end{figure}
        \end{column}
    \end{columns}
\end{frame}

\begin{frame}{Phase B $\sim$ Construction of Initial Landmark Graph}
    \begin{itemize}
        \item Location of landmarks correspond to the \textbf{location of an element} inside the \textbf{floor map};
        \item The \textbf{landmark graph} is constructed starting from \textbf{map information}:
            \begin{itemize}
                \item \textit{Incomplete}: lacks room-level details (e.g., furniture, desks);
                \item Cannot locate WiFi and light landmarks from floor plan.
            \end{itemize}
    \end{itemize}
\end{frame}

\begin{frame}{Phase C $\sim$ Updating of Landmark Graph}
    \begin{itemize}
     
      \item \textbf{Learning New Landmarks}:
        \begin{itemize}
          \item Crowdsourcing: Collect $N$ user trajectories;
          \item Each trajectory contains $n_i$ potential landmarks \{$v_1^i, v_2^i, \dots, v_{n_i}^i$\}.
        \end{itemize}
      \item \textbf{Potential Landmarks}:
        \begin{itemize}
          \item Location points \textbf{satisfying} (at least) one \textbf{detection rule};
          \item \textbf{Not yet included} in the current \textbf{landmark graph}.
        \end{itemize}
      \item \textbf{Updating Algorithm}:
        \begin{itemize}
          \item \textit{Process}: Updates a landmark graph by \textbf{clustering potential landmarks} based on \textit{distance} and \textit{rules}, removing clusters with insufficient elements. Then ads new nodes and edges to the graph based on the cluster centers and detection rules.
        \end{itemize}
    \end{itemize}
\end{frame}