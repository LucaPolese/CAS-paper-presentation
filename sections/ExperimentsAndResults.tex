\section[Results]{Experiments and Results}
\begin{frame}{Experimental Setup}
    \begin{itemize}
        \item \textit{Location}: Office building with eight floors;
        \item \textit{Environment}: Elevators, staircases, corridors, common rooms, office rooms;
        \item \textit{Testing path}: Two floors, approximately 362 meters long;
        \item \textit{Device}: Google Nexus 6 smartphone;
        \item \textit{Sensors}: WiFi, accelerometer, magnetometer, gyroscope, barometer, light sensor;
        \item \textit{Participants}: Six volunteers;
        \item \textit{Task}: Walk preset path, report markers for location accuracy.
    \end{itemize}
\end{frame}

\begin{frame}{Data Collection}
    \begin{block}{How the experient was carried out}
        The participants walked along the preset path with the phone in hand and reported the preset markers they encountered to evaluate the location accuracy.
    \end{block}
    \begin{itemize}
        \item Recorded data:
            \begin{itemize}
                \item \textbf{MAC addresses} of visible WiFi access points and \textbf{RSS};
                \item Sensor readings from accelerometer, gyroscope, compass, barometer, light sensor.
            \end{itemize}
        \item All data timestamped for alignment and inference;
        \item Participants clicked markers on Android app to set the locations.
    \end{itemize}
\end{frame}

\begin{frame}{Step Counting and Step Length Estimation}
    \begin{itemize}
        \item Step counting step detection and step length estimation have an impact on the accuracy of localization: participant was asked to walk 300 steps;
        \item Accuracy of step counting:
            \begin{itemize}
                \item Peak detection: $\sim$ 94\%;
                \item Peak detection with constraints: > 97\%.
            \end{itemize}
        \item Step length estimation:
            \begin{itemize}
                \item Based on stable step length during natural walking;
                \item Step length influenced by walking frequency (step periodicity);
                \item Step periodicity stability: Most steps in [0.5, 0.7] interval.
            \end{itemize}
    \end{itemize}
\end{frame}

\begin{frame}{Initial Location Determination}
    \begin{itemize}
        \item Ten random starting points along the path;
        \item Tested the HMM-based method to initialize th localization system;
        \item \textbf{Results}: Average distance a user has to travel to determine initial location is $\sim$ 9.95 m.
    \end{itemize}
\end{frame}

\begin{frame}{Numerical results}
    \begin{itemize}
        \item \textit{Performance}:
            \begin{itemize}
                \item \textit{Proposed method}: \textbf{88\% accuracy} with \textbf{error <1.5 m};
                \item[]
                \item \textit{Map filtering method}: \textbf{63\% accuracy};
                \item \textit{WiFi fingerprinting method}: \textbf{33\% accuracy};
                \item \textit{PDR methods}: \textbf{Worst performance} due to lack of spatial constraints.
            \end{itemize}
        \item \textit{Mean error}:
            \begin{itemize}
                \item \textit{Proposed method}: \textbf{0.80 m};
                \item[]
                \item \textit{Map filtering method}: \textbf{$\sim$ 1.7 m};
                \item \textit{WiFi fingerprinting method}: \textbf{3.5 m};
                \item \textit{PDR methods}: \textbf{> 7 m}.
            \end{itemize}
        \item \textit{Efficiency} - comparison with \textit{map filtering method}:
            \begin{itemize}
                \item $\sim$ 5 times faster;
                \item Landmark graph-based correction, no wall/obstacle detection;
                \item Suitable for resource-limited platforms.
            \end{itemize}
    \end{itemize}
\end{frame}